The advent of the computer revolutionized the Science and technology vehemently. As a result of this, human thinking and intuition prospered limitlessly to an extent that flying which was thought of as an impossible feat was achieved. The computer became so popular and widely used to an extent that it became a mere necessity in the daily life.  \\

\noindent
As the computer was used in a myriad of applications, techniques  were developed acquire and process data. For this process, data acquisition systems were invented. A data acquisition system (DAQ system) can be simply defined as a system developed to measure an electrical or physical phenomenon such as voltage, current, temperature, pressure, or sound with a computer.\\

\begin{figure}[!h]
	\centering
	\includegraphics[width=0.65\linewidth]{pics/labview/Fig-3-An-Internet-of-Things-based-Data-Acquisition-System}
	\caption{Components of a DAQ system}
	\label{fig:fig-3-an-internet-of-things-based-data-acquisition-system}
\end{figure}


\noindent
With the development of the data acquisition systems, software able to control these systems were introduced. LabVIEW is such a software. LabVIEW (Laboratory Virtual Instrument Engineering Workbench ) is a system-design platform and development environment for a visual programming language. Unlike other programming languages like C or Java, LabVIEW uses a data-flow programming language consisting of icons instead of lines of text.\\

\noindent
When using LabVIEW to acquire and manipulate data, a data acquisition hardware will have to be used to convert the physical parameter which is being measured to a signal which can be processed by the software through the computer. For this purpose, the most commonly used component is the DAQ card. 
\begin{figure}[!ht]
	\centering
	\includegraphics[width=0.6\linewidth]{pics/labview/ucsia}
	\caption{A DAQ system with a DAQ card and LabVIEW}
	\label{fig:ucsia}
\end{figure}


\noindent


